
\documentclass[a4,10pt]{ctexart}

\usepackage{ctex}
\usepackage[utf8]{inputenc}
\usepackage{amsfonts,amsmath,amscd,amssymb,amsthm}
\usepackage{latexsym,bm}
\usepackage{cite}
\usepackage{mathtools,mathdots,graphicx,array}
\usepackage{fancyhdr}
\usepackage{lastpage}
\usepackage{color}
\usepackage{enumitem}
\usepackage{mpdoc}
\usepackage{diagbox}
\usepackage{xcolor,tcolorbox,tikz,tkz-tab,mdframed,tikz-cd}
\usepackage{framed}
\usepackage{verbatim}
\usepackage{extarrows}
\usepackage{fontspec}
\newcommand*{\dif}{\mathop{}\!\mathrm{d}}
\newcommand*{\arsinh}{\mathop{}\!\mathrm{arsinh}}
\newcommand*{\artanh}{\mathop{}\!\mathrm{artanh}}
\newcommand*{\arcosh}{\mathop{}\!\mathrm{arcosh}}
\newcommand*{\Li}{\mathop{}\!\textrm{Li}}



\begin{document}
\pagenumbering{roman}
\title{随机过程课程作业}
\author{202328015926048-丁力}
\date{\today}
\maketitle
\tableofcontents
\newpage
\pagenumbering{arabic}
\newpage

\section{随机过程及其分类}

\begin{ti}{1}{}

    设随机变量 $X$ 服从参数为 1 的指数分布, 随机变量 $Y \sim N(0,1)$, 
    且 $X$ 与 $Y$ 独立, 试求随机变量 $Z=\sqrt{2 X}|Y|$ 的分布密度函数。
    
    \begin{qj}
    % 在此处填写答案
    
    从上我们可以知道, $X$ 服从参数为$\lambda=1$的指数分布, 
    $Y$服从标准正态分布。
    
    那么,我们可以得到两者对应的概率密度函数:

    \begin{align}
        f_{X}(x) &= 
        \begin{cases}
            e^{-x} , x>0 \\
            0, \mbox{其他}
        \end{cases}
                \\
        f_{Y}(y) &= \frac{1}{\sqrt{2\pi}} e^{-\frac{1}{y^2}}, y \in R
    \end{align}

    
    由于$X$和$Y$相互独立, 所以有:

\begin{equation}
    f_{XY}(x,y) = f_X(x) \times f_Y(y) =\frac{1}{\sqrt{2\pi}}  e^{-x} e^{-\frac{1}{y^2}}
\end{equation}

                                                                                                                                                                                                                                                                                                                                                                                                                                                                                                                                                                                                                                                                                                                                                                                                                                                                                                                                                                                                             
    由于$Z=\sqrt{2 X}|Y|$, 结合上面内容,我们可以知道:
    \begin{equation}
    f_Z(z) = 0, z<0
    \end{equation}

    
        

    \end{qj}
    \end{ti}
    
    \begin{ti}{2}{}

    设随机变量 $X_1, X_2$ 独立同分布, 服从参数为 $\lambda>0$ 的指数分布, 试证明随机变量
    \begin{equation}
    \frac{X_1}{X_1+X_2} \sim U[0,1] 
   \end{equation}
    
    \begin{proof}
    % 在此处填写答案
    从上我们可以知道:
    
    \begin{align}
        f_{X_1}(x_1) &=   \begin{cases}
            \lambda e^{-\lambda x_1}, x_1>0 \\
            0, \mbox{其他}\\
        \end{cases} \\
        f_{X_2}(x_2) &=  \begin{cases}
            \lambda e^{-\lambda x_2}, x_2>0 \\
            0, \mbox{其他}\\
        \end{cases}
    \end{align}
    
    \end{proof}
    \end{ti}
    
    \begin{ti}{3}{}

    设随机向量 $(X, Y)$ 的两个分量相互独立, 且均服从标准正态分布 $N(0,1)$。
    
    (a) 分别写出随机变量 $X+Y$ 和 $X-Y$ 的分布密度。
    
    \begin{proof}
    % 在此处填写答案
    \end{proof}
    
    (b) 试问: $X+Y$ 与 $X-Y$ 是否独立?说明理由。
    
    \begin{proof}
    % 在此处填写答案
    \end{proof}
    \end{ti}
    
    \begin{ti}{4}{}

    设二维随机变量 $(X, Y)$ 的联合密度函数为:
    \begin{equation}
    f(x, y)=\left\{\begin{array}{lc}
    24(1-x) y, & 0<y<x<1 \\
    0, & \text { 其它 }
    \end{array}\right.
   \end{equation}
    
    (a) 试求边缘密度函数 $f_X(x)$ 和 $f_Y(y)$, 以及条件密度函数 $f_{X \mid Y}(x \mid y)$ 和 $f_{Y \mid X}(y \mid x)$.
    
    \begin{proof}
    % 在此处填写答案
    \end{proof}
    
    (b) 当 $0<y<1$ 时, 确定 $E\{X \mid Y=y\}$, 以及 $E\{X \mid Y\}$ 的分布密度函数。
    
    \begin{proof}
    % 在此处填写答案
    \end{proof}
    \end{ti}
    


% \section{引言}

% 最近闲来无事, 整理了一个写文章看上去比较好用的模板. 

% 你可以对照源代码, 来看一下如何使用这些内容.

% 需要一定的基础. 

% \section{常用环境}
% \begin{yd}{这是一个约定}{}
              
% 约定的内容.
% \end{yd}

% 上述就是一些文本了. 使用的代码是
% \begin{lstlisting}{language=latex}
% \begin{yd}{这是一个约定}{}
              
% 约定的内容.
% \end{yd}
% \end{lstlisting}

% \begin{zs}
        
% 这是一个注释
    
% \end{zs}
    
% \begin{xt}
        
% 这是一个习题.
    
% \end{xt}
    
% \begin{lt}
        
% 这是一个问题.
    
% \end{lt}

% \begin{yl}
        
% 这是一个引理.
    
% \end{yl}

% \begin{dl}{A}{}
        
% 这是一个定理.
    
% \end{dl}
    
% \begin{tl}{A}{}
        
% 这是一个推论.
    
% \end{tl}

% \begin{dy}{A}{}
        
% 这是一个定义.
    
% \end{dy}

% \begin{jl}{A}{}
        
% 这是一个结论.
    
% \end{jl}

% \begin{mt}{A}{}
        
% 这是一个命题.
    
% \end{mt}

% \begin{ti}{A}{}
        
% 这是一个题目.
    
% \end{ti}

% \begin{cx}{A}{}
        
% 这是一个猜想.
    
% \end{cx}

% \begin{zy}
        
% 这是注意.
    
% \end{zy}

% \begin{ts}
        
% 这是一点提示.
    
% \end{ts}

% \begin{lt}
        
% 这是一个例题.
    
% \end{lt}

% \begin{proof}
% 你还可以加一点证明. 
% \end{proof}

% 我们注意到, 所有的数学公式将自动转换成行间公式的大小, 比如${1\over 2^k}$, $\sum_{i=0}^{998244353}i$. 

% \section{起源与未来的修改计划}

% 起源与hkmod的模板, 添加了一些常用的标志词. 可以在mpdoc.sty里面进行更改, 相信根据注释你也会. 

% 使用愉快! 
    
    
    
    
   

\end{document}
