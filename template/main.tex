
\documentclass[a4,10pt]{ctexart}

\usepackage{ctex}
\usepackage[utf8]{inputenc}
\usepackage{amsfonts,amsmath,amscd,amssymb,amsthm}
\usepackage{latexsym,bm}
\usepackage{cite}
\usepackage{mathtools,mathdots,graphicx,array}
\usepackage{fancyhdr}
\usepackage{lastpage}
\usepackage{color}
\usepackage{enumitem}
\usepackage{mpdoc}
\usepackage{diagbox}
\usepackage{xcolor,tcolorbox,tikz,tkz-tab,mdframed,tikz-cd}
\usepackage{framed}
\usepackage{verbatim}
\usepackage{extarrows}
\usepackage{fontspec}
\newcommand*{\dif}{\mathop{}\!\mathrm{d}}
\newcommand*{\arsinh}{\mathop{}\!\mathrm{arsinh}}
\newcommand*{\artanh}{\mathop{}\!\mathrm{artanh}}
\newcommand*{\arcosh}{\mathop{}\!\mathrm{arcosh}}
\newcommand*{\Li}{\mathop{}\!\textrm{Li}}



\begin{document}
\pagenumbering{roman}
\title{随机过程课程作业}
\author{202328015926048-丁力}
\date{\today}
\maketitle
\tableofcontents
\newpage
\pagenumbering{arabic}
\newpage



\begin{ti}{A}{}
Q
\end{ti}
    


% \section{引言}

% 最近闲来无事, 整理了一个写文章看上去比较好用的模板. 

% 你可以对照源代码, 来看一下如何使用这些内容.

% 需要一定的基础. 

% \section{常用环境}
% \begin{yd}{这是一个约定}{}
              
% 约定的内容.
% \end{yd}

% 上述就是一些文本了. 使用的代码是
% \begin{lstlisting}{language=latex}
% \begin{yd}{这是一个约定}{}
              
% 约定的内容.
% \end{yd}
% \end{lstlisting}

% \begin{zs}
        
% 这是一个注释
    
% \end{zs}
    
% \begin{xt}
        
% 这是一个习题.
    
% \end{xt}
    
% \begin{lt}
        
% 这是一个问题.
    
% \end{lt}

% \begin{yl}
        
% 这是一个引理.
    
% \end{yl}

% \begin{dl}{A}{}
        
% 这是一个定理.
    
% \end{dl}
    
% \begin{tl}{A}{}
        
% 这是一个推论.
    
% \end{tl}

% \begin{dy}{A}{}
        
% 这是一个定义.
    
% \end{dy}

% \begin{jl}{A}{}
        
% 这是一个结论.
    
% \end{jl}

% \begin{mt}{A}{}
        
% 这是一个命题.
    
% \end{mt}

% \begin{ti}{A}{}
        
% 这是一个题目.
    
% \end{ti}

% \begin{cx}{A}{}
        
% 这是一个猜想.
    
% \end{cx}

% \begin{zy}
        
% 这是注意.
    
% \end{zy}

% \begin{ts}
        
% 这是一点提示.
    
% \end{ts}

% \begin{lt}
        
% 这是一个例题.
    
% \end{lt}

% \begin{proof}
% 你还可以加一点证明. 
% \end{proof}

% 我们注意到, 所有的数学公式将自动转换成行间公式的大小, 比如${1\over 2^k}$, $\sum_{i=0}^{998244353}i$. 

% \section{起源与未来的修改计划}

% 起源与hkmod的模板, 添加了一些常用的标志词. 可以在mpdoc.sty里面进行更改, 相信根据注释你也会. 

% 使用愉快! 
    
    
    
    
   

\end{document}
